%%%%%%%%%%%%%%%%%%%%%%%%%%%%%%%%%%%%%%%%%
% Libro Programación FP: guía de supervivencia
% Luis del Moral Martínez
%
% Licencia:
% CC BY-NC-SA 4.0 https:(//creativecommons.org/licenses/by-nc-sa/4.0/)
%
% Este documento usa la plantilla 'The Legrand Orange Book', version 2.4 (26/09/2018)
% La plantilla fue descargada de http://www.LaTeXTemplates.com
%
% Autor original de la plantilla:
% Mathias Legrand (legrand.mathias@gmail.com) con modificaciones realizadas por:
% Vel (vel@latextemplates.com)
%
% Licencia original de la plantilla:
% CC BY-NC-SA 3.0 (http://creativecommons.org/licenses/by-nc-sa/3.0/)
% (DAW y DAM)
%
% Nota importante:
% Las imágenes de cabecera de los capítulos deben tener una relación de aspecto de 2:1 (anchura:altura)
% ejemplo: 920px x 460px (anchura x altura).
%
%%%%%%%%%%%%%%%%%%%%%%%%%%%%%%%%%%%%%%%%%

%----------------------------------------------------------------------------------------
%	PAQUETES Y CONFIGURACIÓN DEL DOCUMENTO
%----------------------------------------------------------------------------------------

\documentclass[12pt,fleqn]{book} % Fuente por defecto y ecuaciones alineadas a la izquierda

\input{structure.tex} % Se inserta el fichero structure.tex que define toda la estructura del documento

%----------------------------------------------------------------------------------------

\begin{document}

%----------------------------------------------------------------------------------------
%	PÁGINA DE TÍTULO
%----------------------------------------------------------------------------------------

\begingroup
\thispagestyle{empty} % Se suprimen los encabezados y pies de página en la página de título
\begin{tikzpicture}[remember picture,overlay]
\node[inner sep=0pt] (background) at (current page.center) {\includegraphics[width=\paperwidth]{background.pdf}};
\draw (current page.center) node [fill=ocre!30!white,fill opacity=0.6,text opacity=1,inner sep=1cm]{\Huge\centering\bfseries\sffamily\parbox[c][][t]{\paperwidth}{\centering Programación en FP\\[15pt] % Título del libro
{\Large Guía de supervivencia}\\[20pt] % Subtítulo
{\huge Luis del Moral Martínez}}}; % Nombre del autor
\end{tikzpicture}
\vfill
\endgroup

%----------------------------------------------------------------------------------------
%	PÁGINA DE COPYRIGHT Y LICENCIAMIENTO
%----------------------------------------------------------------------------------------

\newpage
~\vfill
\thispagestyle{empty}

\noindent \includegraphics[scale=0.08]{green-planet.png}

\noindent Contribuye al ahorro de papel. No imprimas el libro si no es necesario.
\newline

\noindent Copyright \copyright\ 2021 Luis del Moral Martínez
\newline

\noindent Licenciado según la licencia Creative Commons Atribución-NoComercial 4.0 Internacional (CC BY-NC 4.0). Puede obtener una copia de la licencia en:
\url{https://creativecommons.org/licenses/by-nc-sa/4.0/}.
\newline

\noindent Versión 1.0 (febrero de 2021)

%----------------------------------------------------------------------------------------
%	TABLA DE CONTENIDOS
%----------------------------------------------------------------------------------------

\chapterimage{chapter_head_1.pdf} % Imagen de cabecera de la tabla de contenidos

\pagestyle{empty} % Desactiva las cabeceras y los pies de página para todas páginas subsiguientes

\tableofcontents % Imprime la tabla de contenidos

\listoffigures % Imprime la lista de figuras

\listoftables % Imprime la lista de tablas

\cleardoublepage % Fuerza que el primer capítulo empiece en una página impar para que salga en la parte derecha del libro

\pagestyle{fancy} % Habilita de nuevo los encabezados y pies de página

%----------------------------------------------------------------------------------------
%	CONTENIDO DEL LIBRO (PARTES Y CAPÍTULOS)
%----------------------------------------------------------------------------------------

%----------------------------------------------------------------------------------------
%	PARTE I
%----------------------------------------------------------------------------------------
\part{Introducción a la programación}

%----------------------------------------------------------------------------------------
%	CAPÍTULO 1
%----------------------------------------------------------------------------------------
%----------------------------------------------------------------------------------------
%	CAPÍTULO 1
%----------------------------------------------------------------------------------------
\chapterimage{chapter_head_2.pdf} % Imagen del capítulo

\chapter{Presentación}

\section{Te doy la bienvenida}

Si tienes este libro entre manos, o puede que abierto en la pantalla de tu ordenador, quiero darte la bienvenida al que espero que 
sea un curso muy ameno y productivo sobre programación. A lo largo de este libro espero qus desarrollemos técnicas efectivas
para que aprendas a programar aplicaciones informáticas. En este breve capítulo quiero explicarte, en primer lugar, el propósito de esta
obra, ofrecerte una serie de consejos bastante importantes antes de entrar en materia y, por supuesto, explicarte cómo está organizado
el libro. Si ya has programado antes o tienes alguna noción básica sobre programación, puedes echar un vistazo al índice y comenzar
por el capítulo que prefieras, pero yo te recomiendo que empecemos desde el principio. 

\section{Propósito del libro}

Este libro pretende enseñarte a programar desde cero. Es por esto por lo que sus contenidos se han pensado para abarcar todas las fases
de la creación de un programa, suponiendo que no tienes ningún conocimiento previo sobre programación. Así que no te preocupes si no
has programado nunca, y mucho menos si acabas de aterrizar en un ciclo formativo de informática y empiezas a sentir cierto temor sobre
la asignatura de programación. Todos los contenidos que contiene este libro, así como los de las futuras mejoras y versiones, se 
enfocan a las asignaturas de programación del primer curso de los ciclos formativos de grado superior en \textbf{DAM} (Desarrollo de 
Aplicaciones Multiplataforma) y \textbf{DAW} (Desarrollo de Aplicaciones Web). Además, pretenden cumplir con los currículos establecidos en toda
la normativa vigente, teniendo siempre en cuenta el enfoque práctico que tiene la formación profesional, así como la orientación
hacia las prácticas y las necesidades que tienen las empresas hoy en día.

Dicho esto, y espero que con casi todos los miedos desterrados, debo decirte que pretendo ser escueto y práctico en todos los 
conceptos que estudiemos, acompañándote en cada paso y recomendado siempre la mejor estrategia que debes seguir. También voy a evitar
irme por las ramas con definiciones enrevesadas o proponiéndote decenas de enlaces con información adicional. Mi objetivo al escribir
este libro es muy claro, y es que pretendo que puedas superar esta asignatura con los conocimientos y conceptos que vamos a desarrollar
entre estas páginas. En la actualidad hay demasiadas fuentes de información y esto sobrecarga bastante a muchos estudiantes, impidiendo 
que puedan seleccionar la información más útil y descartar la información tediosa o redundante. 

\section{Antes de comenzar}

(Aquí algunos consejos)

\section{Organización del libro}

Antes de que entremos en materia me gustaría explicarte cómo se organiza este libro y detallar algunos de los elementos claves que
usaremos a lo largo de los diferentes capítulos de esta obra. A continuación te detallaré someramente el contenido de cada
una de las partes del resto del libro (como sabes, puedes comenzar por el capítulo que prefieras, pero te aconsejo que empieces 
en orden):

\begin{itemize}
    \item \textbf{Primera parte} {
        \begin{itemize}
            \item \textbf{Capítulo 2}: 
            \item \textbf{Capítulo 3}: 
            \item \textbf{Capítulo 4}: 
        \end{itemize}
        }

\end{itemize}

Otro aspecto importante antes de comenzar es la notación que usaremos en el libro. Esta será bastante sencilla, puesto que vamos
a destacar dos elementos fundamentales que te acompañarán a lo largo de los capítulos: los ejemplos y los consejos. Mientras que
en los ejemplos voy a incluir ciertos códigos o figuras que te ayudarán a comprender mejor los conceptos, en los consejos te detallaré
y explicaré algunos detalles adicionales que considero que no puedes pasar por alto.

\begin{ejemplo}
Esto es un ejemplo de código fuente en C/C++:
\lstset{language=C++,
                showstringspaces=false
                basicstyle=\ttfamily,
                keywordstyle=\color{blue}\ttfamily,
                stringstyle=\color{red}\ttfamily,
                commentstyle=\color{darkgreen}\ttfamily,
                morecomment=[l][\color{magenta}]{\#}
}
\begin{lstlisting}
#include<stdio.h>
#include<iostream>
    
// Esto es un comentario
int main(void)
{
  printf("Imprimiendo por consola\n");
  return 0;
}
\end{lstlisting}
\end{ejemplo}

\begin{consejo}
The concepts presented here are now in conventional employment in mathematics. Vector spaces are taken over the field $\mathbb{K}=\mathbb{R}$, however, established properties are easily extended to $\mathbb{K}=\mathbb{C}$.
\end{consejo}

Finalmente, en las futuras versiones de este libro (y sí hablo de versión y no de \textit{edición} porque pretendo y espero que 
este libro evolucione a lo largo del tiempo, como si de una aplicación informática se tratara) espero incorporar una página web
o un repositorio de \textbf{GitHub}\footnote{GitHub es una empresa que proporciona servicios de alojamiento de repositorios de 
código que emplean la tecnología de control de versiones de Git. No te preocupes ahora por este término. Lo comentaremos
en su debido momento y te enseñaré todo lo que debes saber para hacer uso del mismo.} donde aloje código fuente y 
ejemplos complementarios a los que se recogen en este documento.
 % Se incluye el capítulo 1

%----------------------------------------------------------------------------------------
%	CAPÍTULO 2
%----------------------------------------------------------------------------------------
%----------------------------------------------------------------------------------------
%	CAPÍTULO 2
%----------------------------------------------------------------------------------------
\chapterimage{chapter_head_2.pdf} % Imagen del capítulo

\chapter{¿Dónde me he metido?}

\section{¿En qué consiste programar?}

\section{¿Qué importancia tiene la programación?}

\section{Relación con otras materias}

\section{Ejercicios propuestos}
 % Se incluye el capítulo 2

%----------------------------------------------------------------------------------------
%	CAPÍTULO 3
%----------------------------------------------------------------------------------------
%----------------------------------------------------------------------------------------
%	CAPÍTULO 3
%----------------------------------------------------------------------------------------
\chapterimage{chapter_head_2.pdf} % Imagen del capítulo

\chapter{¿Como funciona un ordenador?}
 % Se incluye el capítulo 3

%----------------------------------------------------------------------------------------
%	CAPÍTULO 4
%----------------------------------------------------------------------------------------
%----------------------------------------------------------------------------------------
%	CAPÍTULO 4
%----------------------------------------------------------------------------------------
\chapterimage{chapter_head_2.pdf} % Imagen del capítulo

\chapter{¿Qué es un algoritmo?}

\section{Concepto de algoritmo}

\section{Características de un algoritmo}

\section{Resolución de problemas}

\section{Creación de algoritmos}

\subsection{Pseudocódigo}

\subsection{Representación gráfica}

\section{Paradigmas de programación}

\section{Lecturas recomendadas}

\section{Software y tipos de software}

\section{Lenguajes de programación}

\section{Ejercicios propuestos}

Completa tu aprendizaje realizando los siguientes ejercicios propuestos:

\begin{exercise}
Bla bla bla
\end{exercise}

\begin{exercise}
Bla bla bla
\end{exercise}

\begin{exercise}
Bla bla bla
\end{exercise}

\begin{exercise}
Bla bla bla
\end{exercise}
 % Se incluye el capítulo 4
 % Se incluye la parte I

%----------------------------------------------------------------------------------------
%	PARTE II
%----------------------------------------------------------------------------------------
\part{Programación en C/C++}
 % Se incluye la parte II

%----------------------------------------------------------------------------------------
%	PARTE III
%----------------------------------------------------------------------------------------
\input{parte3.tex} % Se incluye la parte III

%----------------------------------------------------------------------------------------
%	BIBLIOGRAFÍA
%----------------------------------------------------------------------------------------

\part*{Bibliografia e índice alfabético}

\chapter*{Bibliografía}
\addcontentsline{toc}{chapter}{\textcolor{ocre}{Bibliografía}} % Se añade el encabezado de la bibliografía en el índice de contenidos
 
%------------------------------------------------

\section*{Libros de referencia}
\addcontentsline{toc}{section}{Libros de referencia}
\printbibliography[heading=bibempty,type=book]

%----------------------------------------------------------------------------------------
%	ÍNDICE
%----------------------------------------------------------------------------------------

\cleardoublepage % Nos aseguramos que el índice empiece en una página impar (lado derecho)
\phantomsection
\setlength{\columnsep}{0.75cm} % Fija el espaciado entre las dos columnas del índice
\addcontentsline{toc}{chapter}{\textcolor{ocre}{Índice alfabético}} % Añade el encabezado del índice en el índice de contenidos
\printindex % Imprime el índice

%----------------------------------------------------------------------------------------

\end{document}
