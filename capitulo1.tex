%----------------------------------------------------------------------------------------
%	CAPÍTULO 1
%----------------------------------------------------------------------------------------
\chapterimage{chapter_head_2.pdf} % Imagen del capítulo

\chapter{Presentación}

En este capítulo se introduce el propósito y el contenido de esta obra, detallando cómo está organizada cada parte y capítulo.
Además, se presenta una serie de recomendaciones previas antes de iniciar el resto de capítulos. Se recomienda leer este
capítulo de forma completa antes de pasar al resto de contenidos del libro.

\section{Te doy la bienvenida}

Si tienes este libro entre manos, o puede que abierto en la pantalla de tu ordenador, quiero darte la bienvenida al que espero que 
sea un curso muy ameno y productivo sobre programación. A lo largo de este libro espero que desarrollemos técnicas efectivas
para que aprendas a programar aplicaciones informáticas. En este breve capítulo quiero explicarte, en primer lugar, el propósito de esta
obra, ofrecerte una serie de consejos bastante importantes antes de entrar en materia y, por supuesto, explicarte cómo está organizado
el libro. Si ya has programado antes o tienes alguna noción básica sobre programación, puedes echar un vistazo al índice y comenzar
por el capítulo que prefieras, pero yo te recomiendo que empecemos desde el principio. 

\section{Propósito del libro}

Este libro pretende enseñarte a programar desde cero. Es por esto por lo que sus contenidos se han pensado para abarcar todas las fases
de la creación de un programa, suponiendo que no tienes ningún conocimiento previo sobre programación. Así que no te preocupes si no
has programado nunca, y mucho menos si acabas de aterrizar en un ciclo formativo de informática y empiezas a sentir cierto temor sobre
la asignatura de programación. Todos los contenidos que contiene este libro, así como los de las futuras mejoras y versiones, se 
enfocan a las asignaturas de programación del primer curso de los ciclos formativos de grado superior en \textbf{DAM} (Desarrollo de 
Aplicaciones Multiplataforma) y \textbf{DAW} (Desarrollo de Aplicaciones Web). Además, pretenden cumplir con los currículos establecidos en toda
la normativa vigente, teniendo siempre en cuenta el enfoque práctico que tiene la formación profesional, así como la orientación
hacia las prácticas y las necesidades que tienen las empresas hoy en día.

Dicho esto, y espero que con casi todos los miedos desterrados, debo decirte que pretendo ser escueto y práctico en todos los 
conceptos que estudiemos, acompañándote en cada paso y recomendado siempre la mejor estrategia que debes seguir. También voy a evitar
irme por las ramas con definiciones enrevesadas o proponiéndote decenas de enlaces con información adicional. Mi objetivo al escribir
este libro es muy claro, y es que pretendo que puedas superar esta asignatura con los conocimientos y conceptos que vamos a desarrollar
entre estas páginas. En la actualidad hay demasiadas fuentes de información y esto sobrecarga bastante a muchos estudiantes, impidiendo 
que puedan seleccionar la información más útil y descartar la información tediosa o redundante. 

\section{Antes de comenzar}

Es importante que tengamos en cuenta una serie de recomendaciones antes de que empecemos a programar. Si bien es cierto que repito estos consejos
a mis estudiantes casi todos los días, es posible que algunos te funcionen mejor que otros. Y esto se debe a que no hay dos personas
iguales, desde luego. Es posible que tengas unas dotes innatas para la programación, o puede que te atranques con los primeros enunciados.
Tampoco te preocupes si ese es el caso, porque los consejos que te muestro a continuación los vamos a practicar hasta la saciedad, y
te garantizo que con esfuerzo serás capaz de superar esta asignatura y aprender a programar con bastante soltura:

\begin{itemize}
 \item \textbf{Progreso, no perfección}: cuando se empieza en una nueva disciplina tenemos que evitar pensar en la perfección. Siempre se suele
 comenzar pensando en cómo puedo perfeccionar esta o aquella destreza o qué pasos necesito para llegar a dominar esta tecnología. En mi 
 opinión, esto es un error. Hay que trabajar y esforzarse (sin llegar a perder de vista nuestro objetivo, desde luego). De nuevo, trabajo
 y esfuerzo son los ingredientes básicos para dominar cualquier destreza, y la programación no va a ser menos.
 
 \item \textbf{Programar, programar y programar}: a programar se aprende programando. Igual que a montar en bicicleta se aprende usando la bicicleta. 
 He leído muchas entrevistas a escritores profesionales en las que les pedían que por favor dijeran las claves de su éxito. La mayoría
 siempre respondía: <<a escribir se aprende escribiendo>>.

 \item \textbf{Leer blogs de tecnología y programación}: nos permitirán estar al día sobre las nuevas tendencias en el mundo de la programación. 
 Este consejo es extensible más allá de la tecnología. Es importante leer, y mucho. Como programador, vas a colaborar en el desarrollo
 de aplicaciones que resuelvan el problema de un cliente. Y leer mucho, y de calidad, te permitirá expresarte mejor y comprender mucho
 mejor al cliente. Además, leer abrirá tu mente y te facilitará cualquier proceso de aprendizaje (ya sea en el mundo de la informática o en 
 otro campo de estudio).
 
 \item \textbf{Dibujar los problemas}: este consejo está relacionado con los siguientes capítulos. Es muy importante
 pensar en el problema que nos plantean, entenderlo, y ser capaz de dibujarlo (esto es, como aprenderás dentro de pocas páginas, la capacidad de diseñar un 
 algoritmo\index{Algoritmo}, o un conjunto de pasos, que resuelvan dicho problema). Siempre les pido a mis estudiantes que vengan a los exámenes con folios de sobra,
 lápices, e incluso gomas de borrar. Resulta una técnica muy útil para analizar los ejercicios.
 
 \item \textbf{Leer código de otras personas}: es un buen método para aprender a programar y, por supuesto, para mejorar la calidad de
 nuestro código. De esta forma podemos aprender buenas técnicas de programación. También puede picarte el gusanillo y quizás acabes
 colaborando en un proyecto de \textbf{software libre}\footnote{Es un tipo de software que se distribuye con una licencia
 que permite que los usuarios puedan ejecutarlo, copiarlo, distribuirlo, estudiarlo, modificarlo y mejorarlo. Este libro, por
 ejemplo, se distribuye bajo licencia libre, lo que te permite modificarlo y contribuir a su edición, si lo deseas}, lo que sería estupendo, de hecho.

 \item \textbf{Comentar nuestro código}: los lenguajes de programación (todos ellos) tienen herramientas que permiten documentar el 
 código. Esto permite a los programadores, además de a cualquier persona que se tope con el código en el futuro, comprender el código y
 saber exactamente qué es lo que se está haciendo en cada momento. Llegado el momento te comentaré las mejores técnicas sobre comentarios, y 
 es que hay que adoptar un término medio. No conviene documentar hasta la extenuación; pero es importante incluir comentarios, o de lo contrario
 correrás el riesgo de retomar tu código en el futuro y lo pasarás mal hasta recordar exactamente qué hacía o para qué sirve (créeme, me ha
 pasado unas cuantas veces, y al final se podría haber evitado con comentarios estratégicamente distribuidos por el código).
 
 \item \textbf{Usar un entorno de desarrollo (IDE\index{IDE})}\footnote{Un \textbf{IDE} (Integrated Development Environment) es un entorno de desarrollo
 integrado. Su principal objetivo es el de facilitar la labor de desarrollo de aplicaciones. Algunos de los entornos más populares son
 \textit{Eclipse}, \textit{Visual Studio}, \textit{Visual Studio Code}, \textit{Atom}, \textit{Sublime} o \textit{Notepad++}, entre otros. Muy pronto configurarás tu primer IDE para
 empezar a programar en el lenguaje C/C++.}: los entornos de desarrollo son una de las herramientas indispensables de un programador 
 (te imaginas a un mecánico trabajando sin herramientas). Llegado el momento te enseñaré a configurar y a utilizar los entornos de
 desarrollo \textbf{Visual Studio Code} y \textbf{Eclipse}.
 
 \item \textbf{Revisar y probar el código}: este consejo es importantísimo. En clase y en los exámenes siempre suelo soltar la frase
 <<llevamos mucho tiempo programando, es hora de probar nuestro código>>. Resulta muy importante probar lo que estamos programando, no
 sólo para ver que funciona (también hablaremos sobre los \textbf{errores} y las \textbf{excepciones} y cómo tratarlos; todo a su debido tiempo), sino 
 para comprobar que estamos cumpliendo con el enunciado y que se está resolviendo el problema que se nos pide.
\end{itemize}

A estos consejos hay que sumar siempre una predisposición a practicar y \textit{trastear} con el lenguaje de programación. Es cierto
que esta asignatura puede ser compleja al inicio, pero no hay nada como experimentar con los conceptos recién aprendidos en un 
lenguaje de programación para asentar nuestros conocimientos. Dicho esto, también te recomiendo, además de todo lo anterior, que
procures repasar todos los días o, al menos, hacer pequeños ejercicios donde pongas a prueba todo lo aprendido (ensayos del tipo 
¿qué pasaría si hago esto? o ¿qué sucede si hago el programa de esta forma?).

\section{Organización del libro}

Antes de que entremos en materia me gustaría explicarte cómo se organiza este libro y detallar algunos de los elementos claves que
usaremos a lo largo de los diferentes capítulos de esta obra. A continuación te detallaré someramente el contenido de cada
una de las partes del resto del libro (como sabes, puedes comenzar por el capítulo que prefieras, pero te aconsejo que empieces 
en orden):

\begin{itemize}
    \item \textbf{Primera parte} {
        \begin{itemize}
            \item \textbf{Capítulo 2. ¿Dónde me he metido}: en este capítulo hablaremos de qué significa programar y, sobre todo, 
            qué es lo que se espera de ti en esta asignatura y en un ciclo de desarrollo de aplicaciones (\textbf{DAM} o \textbf{DAW}).
            \item \textbf{Capítulo 3. ¿Cómo funciona un ordenador?}: este capítulo te ayudará a comprender los procesos básicos que
            permiten que funcione un ordenador, entrando únicamente en los detalles que necesitas comprender para aprender a programar.
            \item \textbf{Capítulo 4. ¿Qué es un algoritmo}: detalla los pasos que hay que llevar a cabo para analizar un problema y
            crear un algoritmo que lo resuelva. Este capítulo incorpora un conjunto de herramientas muy útiles para enfrentarse a
            cualquier problema futuro. Es recomendable que vuelvas sobre tus pasos y regreses a este capítulo cuando sea preciso.
        \end{itemize}
        }

\end{itemize}

Otro aspecto importante antes de comenzar es la notación que usaremos en el libro. Esta será bastante sencilla, puesto que vamos
a destacar dos elementos fundamentales que te acompañarán a lo largo de los capítulos: los \textbf{ejemplos} y los \textbf{consejos}. Mientras que
en los \textbf{ejemplos} voy a incluir ciertos códigos o figuras que te ayudarán a comprender mejor los conceptos, en los \textbf{consejos} te detallaré
y detalles adicionales y buenas prácticas que no puedes pasar por alto.

\begin{ejemplo}
Esto es un ejemplo de código fuente en C/C++:
\lstset{language=C++,
                showstringspaces=false,
                columns=fullflexible,
                basicstyle=\ttfamily,
                keywordstyle=\color{blue}\ttfamily,
                stringstyle=\color{red}\ttfamily,
                commentstyle=\color{darkgreen}\ttfamily,
                morecomment=[l][\color{magenta}]{\#}
}
\begin{lstlisting}
#include<stdio.h>
#include<iostream>
    
// Esto es un comentario
int main(void)
{
  printf("Imprimiendo por consola\n");
  return 0;
}
\end{lstlisting}
\end{ejemplo}

\begin{consejo}
Las palabras reservadas de un lenguaje siempre deben escribirse en minúsculas. Ejemplo: \textbf{int}, \textbf{void}, \textbf{for}, \textbf{if}...
\end{consejo}

En las futuras versiones de este libro (y sí hablo de versión y no de \textit{edición} porque pretendo y espero que 
este libro evolucione a lo largo del tiempo, como si de una aplicación informática se tratara) espero incorporar una página web
o un repositorio de \textbf{GitHub}\index{GitHub}\footnote{GitHub es una empresa que proporciona servicios de alojamiento de repositorios de 
código que emplean la tecnología de control de versiones de \index{Git}. No te preocupes ahora por este término. Lo comentaremos
en su debido momento y te enseñaré todo lo que debes saber para hacer uso del mismo.} donde aloje código fuente y 
ejemplos complementarios a los que se recogen en este documento.
