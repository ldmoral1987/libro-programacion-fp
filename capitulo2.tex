%----------------------------------------------------------------------------------------
%	CAPÍTULO 2
%----------------------------------------------------------------------------------------
\chapterimage{chapter_head_2.pdf} % Imagen del capítulo

\chapter{¿Dónde me he metido?}

En este capítulo se explica en qué consiste programar y la importancia que tiene la programación en nuestra sociedad. 
También se relaciona la asignatura de programación con otras materias de \textbf{Desarrollo de Aplicaciones Multiplataforma} (DAM)
y \textbf{Desarrollo de Aplicaciones Web} (DAW), de forma que se comprenda la importancia que tiene aprender a programar de forma correcta
antes de abordar contenidos o materias más complejas de ambos ciclos formativos.

\section{¿En qué consiste programar?}



\section{¿Qué importancia tiene la programación?}



\section{Relación con otras materias}

La programación de aplicaciones está relacionada casi con todas las materias de los ciclos formativos de 
\textbf{Desarrollo de Aplicaciones Multiplataforma} (DAM) y \textbf{Desarrollo de Aplicaciones Web} (DAW), puesto que el 
propósito de ambos ciclos formativos es el de capacitar a los estudiantes para que desarrollen aplicaciones software
(ya sean multiplataforma\footnote{Que pueden ejecutarse de forma compatible en diferentes sistemas hardware} u orientadas a la web).

Es por esto por lo que saber programar, como adelantábamos en la sección anterior, es de vital importancia en la actualidad y,
de hecho, muchos expertos sostienen que el aprendizaje de técnicas de programación debería comenzar mucho antes, incluso en
bachillerato o puede que en la ESO. Y es que crear una aplicación informática no sólo consiste en <<picar código>>, o en crear una \textbf{interfaz de usuario} muy bonita.
Consiste en un proceso que abarca desde la comprensión de los requisitos del cliente hasta la especificación y el diseño de una 
solución, su implementación, testeo y posterior despliegue. 

Por otra parte, cuando los estudiantes llegan al clase siempre creen que van a empezar a desarrollar videojuegos y aplicaciones
alucinantes desde el primer día. Pero siento decirte que esto no es posible sin sentar unas bases sólidas sobre la programación
estructurada y la programación orientada a objetos, proceso que comenzaremos en la segunda y tercera parte de este libro, 
respectivamente. En los siguientes capítulos hablaremos sobre cómo funciona un ordenador y el concepto de \textbf{algoritmo}. Te
recomiendo que continúes leyendo antes de pasar a la acción en la parte 2, con los lenguajes de programación C y C++, puesto
que aprenderás los conceptos básicos sobre el funcionamiento de un ordenador y técnicas de resolución de problemas usando
algoritmos, lo que te aportará el conocimiento suficiente para empezar a escribir tus primeros programas informáticos.

\section{Ejercicios propuestos}

Completa tu aprendizaje realizando los siguientes ejercicios propuestos:

\begin{exercise}
Visualiza los siguientes videos de Youtube y reflexiona sobre las necesidades de programación que existen en la actualidad:

\begin{itemize}
 \item \href{https://www.youtube.com/watch?v=X5Wkp1gsNik}{Todo el mundo debería saber programar}
 \item \href{https://www.youtube.com/watch?v=wldGsJTPJ1o}{¿Por qué hay que aprender a programar?}
\end{itemize}
\end{exercise}

\begin{exercise}
Investiga en los siguientes portales de búsqueda de empleo acerca de la cantidad y número de ofertas de trabajo que existen
en la actualidad relacionadas con la programación (deberás investigar también acerca de los diferentes roles o puestos
que se ofertan relacionados con la programación de aplicaciones):

\begin{itemize}
 \item \href{https://www.tecnoempleo.com/}{Tecnoempleo}
 \item \href{https://www.infojobs.net/}{Infojobs}
 \item \href{https://www.linkedin.com/}{Linkedin}
\end{itemize}
\end{exercise}
